% !TEX root = USEGUIDE.tex

\part{CLASS : General overview}
\chapter{Generalities}
\section{Basic unit\label{sec:unit}}
All time in CLASS should be written in second. It corresponds to the cSecond, a CLASS c++ type, which are a \textbf{long long int} going, in 32 bits \textbf{and} 64 bits, up to  $(2^{63} - 1)~\textrm{s}~ \sim 2.9\cdot 10^{11}$ years, enough for any electro-nuclear scenarios one can consider....\\
 \section{CLASS working process principle}
image : sh�ma de principe de class
\chapter{Facilities descriptions}
All the facility in CLASS project are regroup inside a large group called CLASSFacility (and inherit of all the properties of the CLASSFacility in a C++ way). Inside the CLASSFacility, 3 different types has be defined, the reactor, the FabricationPlant (or more generally, all the fuel cycle front-end facilities) and the backend facilities. 
\section{CLASSFacility\label{sec:CLASSFacility}}
The CLASSFacility should never be used directly in the main CLASS program (the one made to perform the simulation).  The aim of these object is to regroup all the common properties of the nuclear facilities, such as common variables, methods, and builder. 
Its includes 3 variables needed by CLASS environment:
\begin{center}
\begin{minipage}{\textwidth}
\begin{lstlisting}
private :
	int		fId;						//!< Identity of the Facility inside the Parc
	int		fFacilityType;	///< Type of facility :
												/// \li 4 reactor,
												/// \li 8 Pool,
												/// \li 16 FabricationPlant.
	Scenario*		fParc;		//!< Pointer to the main Parc
\end{lstlisting}
\end{minipage}
\end{center}
The \textit{fId} variable correspond to the unique identity number allowing to differentiate all the facility of a certain type. The \textit{fFacilityType} variable correspond to an identity number allowing to separate the different type of facilities. Those to variable are "\textbf{private}", which mean it is not possible to access to them directly, one \textbf{must} to use there Get-xxx() and the Set-xxx() function (even inside the daughter class such as reactor...). \textit{fParc} is a pointer to the main park, which allow to access to the rest of the park. Be careful the \textit{fParc} pointer main not be assigned if the facility is not included in a park...\\
\\
\\
\\
The CLASSFacility also includes all the generic variable and method for time and simple fuel management:
\begin{center}
\begin{minipage}{\textwidth}
\begin{lstlisting}
protected :
	bool		fIsStarted;			///< True if Running, False Otherwise
	bool		fIsShutDown;		///< True if the facility is stoped, False Otherwise
	bool		fIsShutDown;	///< True if Reaching the End of a Facility Cycle

		
	cSecond		fInternalTime;		///< Internal Clock
	cSecond		fInCycleTime;			///< Time spend since the beginning of the last Cycle
	cSecond		fCycleTime;				///< Cycle duration Time

	IsotopicVector	fInsideIV;	///< All IV in the Facility (fuel for reactor, total for all others...)
	IsotopicVector	fCumulativeIVIn;	///< All IV in the Facility (fuel for reactor, total for all others...)
	IsotopicVector	fCumulativeIVOut;	///< All IV in the Facility (fuel for reactor, total for all others...)

		//********* Internal Parameter *********//
private :
	[...]
	cSecond		fCreationTime;		///< CLASS Universal Time of Creation
	cSecond		fLifeTime;				///< Time of life Of the Reactor (Operating's Duration)

\end{lstlisting}
\end{minipage}
\end{center}
\textit{fIsStarted}, \textit{fIsShutDown}, \textit{fIsShutDown} allows to the Dump() method to determine the state of the facility and do what is necessary.\\
\textit{fInternalTime}, \textit{fInCycleTime}, \textit{fCycleTime} variable allows the \hyperref[sec:unit]{time} the time management inside the facility. \textit{fInternalTime} correspond to the last time until the evolution of the facility as been calculated. \textit{fCycleTime} the time length of a cycle in this facility (note that it could be irradiation cycle in a \hyperref[sec:reactor]{Reactor}, fabrication time in a \hyperref[sec:FP]{FabricationPlant}, or cooling time in a \hyperref[sec:pool]{Pool}). And \textit{fInCycleTime} correspond to the time already past in the current cycle.\\
\textit{fCreationTime}, \textit{fLifeTime} are \textbf{private} and used to define the creation time of the facility, and its operation time length.
 
Besides all the time management of the facility, it also contain the basic tool for fuel management the three \hyperref[sec:IV]{IsotopicVector} : \textit{fInsideIV}, \textit{fCumulativeIVIn}, \textit{fCumulativeIVOut}, which correspond respectively to the isotopic vector present in the facility, to the cumulative income inside the facility, and the cumulative outcome outside the facility.

By default af the Get-xxx() and Set-xxx() method associated to those variable are define, and some can be overloaded.

The CLASSFacility also comes with 2 virtual method (which means one must  overloading it when defining a new facility) : \textit{Evolution(cSecond t)} and \textit{Dump()}. They are both used to perform the evolution of the facility. Where the first one (\textit{Evolution}) is used to specify the change inside the facility (mainly fuel evolution), the second one (\textit{Dump}) is used to deal the exchange between facility, such as refilling a reactor or send a fuel to the stock at the end of cooling.

\section{Reactor\label{sec:reactor}}
\subsection{Generalities}
 The aim of this class is to deal the evolution of the fuel inside a reactor.
 The fuel state of the reactor is describe in the IsotopicVector fInsideIV (which are inherit from the  \hyperref[sec:CLASSFacility]{CLASSFacility} class). Its evolution is \textbf{always} contain in the \hyperref[sec:EvolutionData]{EvolutionData} \textit{fEvolutionDB}.\\
  There are 2 way to provide the \hyperref[sec:EvolutionData]{EvolutionData} to the reactor. In the case of fixed fuel the user need to provide it, using the appropriated constructor, the set function, or a \hyperref[sec:CLASSFuelPlan]{CLASSFuelPlan}. In the case of recycled fuel or unfixed fuel, the user need to provide a \hyperref[sec:PhysicsModels]{PhysicsModels}, using the appropriated constructor, the set function, and/or a \hyperref[sec:CLASSFuelPlan]{CLASSFuelPlan}.

\subsection{Constructor\label{sec:reactor:constructor}}
There are many ways to define a reactor.
\subsubsection{Normal constructor}
\begin{center}
\begin{minipage}{\textwidth}
\begin{lstlisting}
	Reactor();		///< Normal Constructor.
\end{lstlisting}
\end{minipage}
\end{center}
Just define a simple reactor without fuel, starting time,  mass of fuel, time of life or anything.
\begin{center}
\begin{minipage}{\textwidth}
\begin{lstlisting}
	Reactor(CLASSLogger* log);
\end{lstlisting}
\end{minipage}
\end{center}
Just define a simple reactor without fuel, starting time,  mass of fuel, time of life or anything, but set a \hyperref[sec:logclass]{CLASSLogger} \textit{log} to send the CLASS message (see section~\ref{sec:logclass}).
\vspace{3cm}

\subsubsection{Fixed fuel constructor}
Constructor defining a reactor using fixed fuel. 
\begin{center}
\begin{minipage}{\textwidth}
\begin{lstlisting}
	Reactor(CLASSLogger* log, EvolutionData evolutivedb, CLASSBackEnd* CBE,
		cSecond creationtime, cSecond lifetime,
		double power, double HMMass, double BurnUp, double CapacityFactor = 1);
\end{lstlisting}
\begin{lstlisting}
	Reactor(CLASSLogger* log, EvolutionData evolutivedb, CLASSBackEnd* Pool,
		cSecond creationtime, cSecond lifetime,
		cSecond cycletime, double HMMass, double BurnUp);
\end{lstlisting}
\end{minipage}
\end{center}
The evolution of the fuel is given by the \hyperref[sec:EvolutionData]{EvolutionData} \textit{evolutivedb} the mass of heavy metal in the reactor core by \textit{HMMass}, the power by \textit{Power} and  its capacity factor by \textit{CapacityFactor}.\\
To avoid mixing between constructor, only 2 constructors exist to set the triplet (power, burnup, cycle time). (if more 2 constructor would have the same number of input variables.)\\
After irradiation the fuel goes in the \hyperref[sec:CLASSBackEnd]{CLASSBackEnd} CBE, which as seen section \label{sec:CLASSBackEnd} can be any CLASSBackEndFacility (SeparationPlant, Pool or storage).

\subsubsection{Reprocessed fuel constructor}
As well as in the fixed fuel constructor, the mass of heavy metal in the reactor core by \textit{HMMass}, the power by \textit{Power} and  its capacity factor by \textit{CapacityFactor}. Only 2 constructors exist to set the triplet (power, burnup, cycle time).\\
The \hyperref[sec:PhysicsModels]{PhysicsModels} is \textit{fuelDB}, and the \hyperref[sec:FabricationPlant]{FabricationPlant} is \textit{FabricationPlant}.
\begin{center}
\begin{minipage}{\textwidth}
\begin{lstlisting}
	Reactor(CLASSLogger* log, PhysicsModels 	fueltypeDB,
		FabricationPlant* fabricationplant, CLASSBackEnd* Pool,
		cSecond creationtime , cSecond lifetime, cSecond cycletime,
		double HMMass, double BurnUp);
\end{lstlisting}
\begin{lstlisting}
	Reactor(CLASSLogger* log, PhysicsModels 	fueltypeDB,
		FabricationPlant* fabricationplant, CLASSBackEnd* Pool,
		cSecond creationtime , cSecond lifetime,
		double Power, double HMMass, double BurnUp, double CapacityFactor);
\end{lstlisting}
\end{minipage}
\end{center}

\subsection{Main methods}
Despite all the Get-xxx() and Set-xxx() method, Reactor possess 3 main method, Evolution(), Dump() and SetStorage() (yes this is a Set-xxx() method...).
\subsubsection{Evolution}
The evolution function is quite simple, it only follow the \hyperref[sec:EvolutionData]{EvolutionData}, containing all the information about the fuel evolution.
When called the \textit{Evolution(cSecond t)} method, the reactor will perform its evolution until the time \textit{t}.\\\\
\fbox{
\begin{minipage}{\textwidth}
\textbf{Note that, the reactor \textcolor{red}{MUST} have a time step at the exact second of its end of cycle, start or shutdown, if not the CLASS will break returning : }\\
"Evolution is too long! There is a problem in Reactor evolution at ..."\\
\textbf{as an error, giving also some time information about the break.}
\end{minipage}
}\\\\
Beside the verification if the reactor is started, and its start, the evolution can encounter 2 cases : if the evolution stop during and cycle to at the end of a cycle.
\begin{center}
\begin{minipage}{\textwidth}
\begin{lstlisting}
if( EvolutionTime + fInCycleTime == fCycleTime )		//End of Cycle
{
	fIsAtEndOfCycle = true;
	fInternalTime += EvolutionTime; 				// Update Internal Time
	fInCycleTime += EvolutionTime;					// Update InCycleTime
	if(t >= GetCreationTime() + GetLifeTime()) // if the Next Cycle don't exist
		fIsShutDown = true;
}
else if(EvolutionTime + fInCycleTime < fCycleTime )			// During Cycle
{
	fInternalTime += EvolutionTime;					// Update Internal Time
	fInCycleTime += EvolutionTime;					// Update InCycleTime

	fInsideIV = fEvolutionDB.GetIsotopicVectorAt( (cSecond)(fInCycleTime/fEvolutionDB.GetPower()*fPower) );	// update the fuel composition
	if(t>=GetCreationTime() + GetLifeTime())	fIsShutDown = true;
}
else
{
	// Evolution goes after the end of cycle.... check it
	exit(1);
}
\end{lstlisting}
\end{minipage}
\end{center}
In the first case, the reactor will only update its \textit{fInsideIV} to the corresponding state.
In the second case, only flag the end of cycle, the rest of the job is don in the \textit{Dump()} method.\\
\\
\\

\subsubsection{Dump}
This method deal all the exchange with the other facility of the scenario. It is divided in two part. The first one trash the used fuel, detecting which case it is (normal end of a cycle, end of cycle ad reactor shutdown, or shutdown during a cycle).
\begin{center}
\begin{minipage}{\textwidth}
\begin{lstlisting}
if(fIsAtEndOfCycle  && !fIsShutDown )
{
	if(fIsStarted  )					// A Cycle has already been done
	{
		fOutBackEndFacility->AddIV(fInsideIV);
		AddCumulativeIVOut(fInsideIV);
	}
	else fIsStarted = true;					// Just start the first cycle

}
else if (fIsAtEndOfCycle  && fIsShutDown )			//shutdown at end of Cycle
{

	fOutBackEndFacility->AddIV(fIVOutCycle);
	AddCumulativeIVOut(fIVOutCycle);
	fInsideIV.Clear();
	fInCycleTime = 0;
	fIsStarted = false;					// shut down the Reactor
}
else if (!fIsAtEndOfCycle && fIsShutDown ) 			//shutdown during Cycle
{
	fOutBackEndFacility->AddIV(fInsideIV);
	AddCumulativeIVOut(fInsideIV);
	fInsideIV.Clear();
	fInCycleTime = 0;
	fIsStarted = false;					// shut down the Reactor
}

\end{lstlisting}
\end{minipage}
\end{center}

The second part correspond to refilling of the reactor...
For the fixed fuel :
\begin{center}
\begin{minipage}{\textwidth}
\begin{lstlisting}
fIsAtEndOfCycle = false;
GetParc()->AddGod(fIVInCycle);
fInsideIV  = fIVBeginCycle;
AddCumulativeIVIn(fIVBeginCycle);
fInCycleTime = 0;
\end{lstlisting}
\end{minipage}
\end{center}
where the fuel material is taken outside of the scenario.\\
\\
\begin{center}
\begin{minipage}{\textwidth}
Or for the reprocessed fuel :
\begin{lstlisting}
fIsAtEndOfCycle = false;
SetNewFuel(fFabricationPlant->GetReactorEvolutionDB(GetId()));
fFabricationPlant->TakeReactorFuel(GetId());
fInsideIV  = fIVBeginCycle;
AddCumulativeIVIn(fIVBeginCycle);
fInCycleTime = 0;
\end{lstlisting}
\end{minipage}
\end{center}
where the fuel and the corresponding EvolutionData is builded by a FabricationPlant using the PhysicsModel.

\subsubsection{SetStorage}
The SetStorage() method must be use with caution !! Because its use will modify the normal operation of the reactor.
It can be used only in the case of a fixed fuel reactor and not stock management (all stock ar mixed inside the storage), when one want to take the material inside the stock instead of outside of the scenario. In this case, an isotopic separation of the provided stock is done, the needed isotope are taken from the stock, the missing part is provided by the OutSide of the scenario. The refilling part of the Dump() method becomes :
\begin{center}
\begin{minipage}{\textwidth}
\begin{lstlisting}
fIsAtEndOfCycle = false;
IsotopicVector BuildIVtmp ;
IsotopicVector GodPart;
//Get The Storage Compostion
BuildIVtmp.Add(fStorage->GetInsideIV().GetIsotopicQuantity());
//Get the rest after IVIn creation
BuildIVtmp -= fIVInCycle;
//Get the God part form this rest
GodPart.Add(BuildIVtmp.GetIsotopicQuantityNeeded()) ;
//Take what you can from Storage...
fStorage->TakeFromStock( fIVInCycle - GodPart);
//And Get the rest from God
GetParc()->AddGod(GodPart);
fInsideIV  = fIVBeginCycle;
AddCumulativeIVIn(fIVBeginCycle);
fInCycleTime = 0;
\end{lstlisting}
\end{minipage}
\end{center}
Again this method is really tricky to use and need to be used with great caution!!!

\subsection{CLASSFuelPlan}
If you don't want to change the fuel type (from a fixed fuel to an other, a fixed fuel to a reprocessed one, changing the final burnup of a fuel...) inside a reactor during its operation stay away from this part.\\
The aim of this class is to allow user to change the fuel parameter during the operation of the reactor, by example setting a reactor with a UOX fuel enriched at 3\% for 3 cycles, then changing to 5\% UOX or even to MOX. All fuel in a CLASSFUelPLan are associated to a burnup. The reactor updating his cycle time to this burnup (considering the power remain the same in a reactor during all its life).\\
The usage of the CLASSFuelPlan is quiet simple... On just need to add to an existing CLASSFuelPlan, a time (to perform the fuel switch), a fuel and an associated burnup. One should note that the fuel in the reactor only change at the end of a cycle (finish the current cycle even after a date to change fuel type.)\\
\begin{center}
\begin{minipage}{\textwidth}
\begin{lstlisting}
void AddFuel(cSecond time,  EvolutionData* fuel, double BurnUp)
void AddFuel(cSecond time,  PhysicsModels* fuel, double BurnUp)
\end{lstlisting}
\end{minipage}
\end{center}

For the one who want to go further, when declaring a reactor (using any type of fuel) (see \ref{sec:reactor:constructor}), a CLASSFuelPlan is created inside the reactor. So it is possible to create a reactor using the normal reactor creator (which include a fuel...) and add after some fuel to the plan using \textit{CLASSFuelPlan*	GetFuelPlan()} method of the reactor, then the AddFuel method of CLASSFuelPlan.\\
ie : 
\begin{center}
\begin{minipage}{\textwidth}
\begin{lstlisting}
CLASSLogger *MyLogger = new CLASSLogger("CLASS_OUTPUT.log",1,2);
EvolutionData* STD900 = new EvolutionData(MyLogger, "STD900.dat",false);
EvolutionData* GARANCE = new EvolutionData(MyLogger, "GARANCE.dat",false);
Reactor* MyReactor = new Reactor(MyLogger, STD900, Cooling_UOX, 0., 40*3600*24*365.25, 900e6, 106, 33, 0.8);
Myreactor->GetFuelPlan()->AddFuel(2001*3600*24*365.25, GRANACE, 42);
\end{lstlisting}
\end{minipage}
\end{center}
If one want to switch between fixed fuel (e.g. PWR UOX) to recycled fuel (e.q. PWR MOX) one need to define a FabricationPlant to the reactor :
\begin{center}x
\begin{minipage}{\textwidth}
\begin{lstlisting}
IM_RK4 *IMRK4 = new IM_RK4(MyLogger);
EQM_LIN_PWR_MOX* EQMLINPWRMOX = new EQM_LIN_PWR_MOX(MyLogger, "EQ_Lin.dat");
EQM_QUAD_PWR_MOX* EQMQUADPWRMOX = new EQM_QUAD_PWR_MOX(MyLogger, "DB.dat");
PhysicsModels* PHYMOD = new PhysicsModels(XSMOX, EQMQUADPWRMOX, IMRK4);
Myreactor->GetFuelPlan()->AddFuel( MOXTime, PHYMOD, 35);
Myreactor->SetFabricationPlant(FP_MOX);

\end{lstlisting}
\end{minipage}
\end{center}




\section{CLASSBackEnd}
The CLASSBackEnd class is a master class which aims to regroup all common properties of the fuel back-end facilities. All other back-end facilities in CLASS inherit of the CLASSBackEnd.\\
In CLASS, a CLASSBackEnd does not control its upstream. Its incoming material flux is pushed by its upstream facility (a Reactor, or an other CLASSBackEnd). It only controls its downstream flux.\\
\textbf{This object is not supposed to be used explicitly in a CLASS input.}
\subsection{Storage}
Storage is a CLASSBack end without associated downstream factory. All the incoming material are stored individually. During the storage, the depletion by decay is taken into account. The storage has to be defined as follow :
\begin{center}
\begin{minipage}{\textwidth}
\begin{lstlisting}
Storage *Stock = new Storage(aCLASSLogger);
\end{lstlisting}
\end{minipage}
\end{center}
 
 
\subsection{Pool\label{sec:pool}}
Pool is a CLASSBack end with an associated downstream factory. All incoming material will be pushed in the downstream factory after a set time, so called CoolingTime. All the incoming material are stored individually. During the cooling process, the depletion by decay is taken into account. The storage has to be defined as follow :
\begin{center}
\begin{minipage}{\textwidth}
\begin{lstlisting}
Storage *Stock = new Storage(aCLASSLogger);
Pool *MyPool = new Pool(aCLASSLogger,Stock, 5*365.25*24.*3600);
\end{lstlisting}
\end{minipage}
\end{center}
In the previous example, a 5 years cooling time has been used.
If no downstream facility is set, all the material will be pushed after cooling to the WASTE of the Scenario. To do so :
\begin{center}
\begin{minipage}{\textwidth}
\begin{lstlisting}
Pool *MyPool = new Pool(aCLASSLogger, 5*365.25*24.*3600);
\end{lstlisting}
\end{minipage}
\end{center}

\subsection{SeparationPlant\label{sec:sepplant}}


\section{Fabrication Plant\label{sec:FP}}

\chapter{Other objects}
\section{ZAI\label{sec:ZAI}}
The ZAi object represents a nucleus, from its charge number, mass number and isomeric state.\\
The object save  the charge number Z, the mass number A and the isomeric state I  of a nucleus : I=0 for ground state , I=1 for the first isomeric state ...\\
To declare a ZAI object proceed as follow :
\begin{center}
\begin{minipage}{\textwidth}
\begin{lstlisting}
ZAI U238 = ZAI(92, 238, 0);
\end{lstlisting}
\end{minipage}
\end{center}
This class includes the mains logical comparators (\emph{e.g} ==, >, !=). Fill free to read the doxygen for more details on the methods associated to this class. (\emph{e.g} A(), Z(), I(), N()...) \cite{doxygen}.
%\vspace{4cm}

\section{IsotopicVector\label{sec:IV}}
\subsection{Generality}
The IsotopicVector object is a collection of ZAI, for each ZAI a number of nuclei is associated (IsotopicVector is a c++ map of ZAI and double, which corresponds to a sorted array of ZAI and its quantity).\\
Two pincipales operation have been defined on IsotopicVector. 
The following illustrates the possible operation allowed for IsotopicVectors :
\paragraph{Definiton \& Addition of nuclei}
\begin{center}
\begin{minipage}{\textwidth}
\begin{lstlisting}
IsotopicVector IV_1;
IsotopicVector IV_2;

IV_1 += 23 * ZAI(92, 238, 0); // Add 23 nucleus of uranium 238 to ZAI_1
IV_1 += 52 * ZAI(92, 235, 0); // Add 52 nucleus of uranium 235 to ZAI_1
\end{lstlisting}
\end{minipage}
\end{center}
\paragraph{Multiplication}
\begin{center}
\begin{minipage}{\textwidth}
\begin{lstlisting}
IV_1 *= 100; // Multiply all the nuclei quantities by 100 -> resulting : 2300 uranium 238 and 5200 uranium 235

IV_2 = IV_1 * 10; // IV_2 will be equal to 10 IV_1 
\end{lstlisting}
\end{minipage}
\end{center}
\paragraph{Sum}
\begin{center}
\begin{minipage}{\textwidth}
\begin{lstlisting}
IsotopicVector IV_sum = IV_1 + IV2; // IV_sum will be equal to 11 IV_1
\end{lstlisting}
\end{minipage}
\end{center}

Some additional operations have been also implemented, such as subtraction. It works as the sum, but if the result of the subtraction is negative for some nuclei, those nuclei are set to zero and the difference is added to the, so called, \textit{fIsotopicQuantityNeeded}. If so, a WARNING will be written on the terminal.\\
@@ Link WARNING \\
\textbf{To insure the quality and the reliability of the simulation, the fIsotopicQuantityNeeded MUST remain empty.}


\subsection{Print method}
You can use the Print() method to write the composition of an IsotopicVector.
When printing the IsotopicVector composition present nuclei, as well as the \textit{needed} one, are written with their corresponding quantity (unit: nucleus number).\\


\subsection{GetTotalMass}
Return the mass of the IsotopicVector in tons using :
\begin{center}
\begin{minipage}{\textwidth}
\begin{lstlisting}
double TotalMass = IV.GetTotalMass();
\end{lstlisting}
\end{minipage}
\end{center}


\subsection{Multiplication between IsotopicVector}
The result of this operation is an IsotopicVector, where each nucleus quantity is the product of the corresponding nucleus quantity of the two IsotopicVector.\\
In other words :\\
If a nucleus A is present in both IsotopicVector, with respective quantity $\alpha$ and $\beta$, the resulting IsotopicVector will contain $\alpha\times\beta$ nucleus A. If the nucleus A is not present in both IsotopicVector, the resulting IsotopicVector will not contain the nucleus A.\\

\textit{By exemple, this method can be used to apply separation efficiency: one IsotopicVector containing real material and the other one containing separation efficiency of each nucleus.}

\section{Log management : CLASSLogger}\label{sec:logclass}
In CLASS, all messages are handled by the CLASSLogger object. There are 4 verbose levels, see table~\ref{tab:verblevel}.

\begin{table}[H]
\begin{center}
\caption{Verbose levels}
\label{tab:verblevel}
\begin{tabular}{|c|c|l|}
\hline
level \# & meaning & informations\\
\hline
0 & ERROR & This is the default. \textit{It makes the code to stop}\\
\hline
1 & WARNING & LVL 0 + something may go wrong but the code continue running\\
\hline
2 & INFO &  LVL 1 + simple informations about ongoing process \\
\hline
3 & DEBUG & LVL 2 + each method begin and end \\
\hline
\end{tabular}
\end{center}
\end{table}

There are two outputs for these messages : the standard output (terminal) and a logfile. For each output a verbose level can be assigned as follow :
\begin{center}
\begin{minipage}{\textwidth}
\begin{lstlisting}
CLASSLogger *Logger = new CLASSLogger("CLASS_OUTPUT.log",1,2);  
\end{lstlisting}
\end{minipage}
\end{center}
In the preceding example, verbose level 1 (WARNING) has been set for the terminal output and level 2 (INFO) for the second output which is the logfile named CLASS\_OUTPUT.log. 

\section{EvolutionData}\label{sec:EvolutionData}
An EvolutionData aims to describe the evolution of an IsotopicVector through a physical process (decay or irradiation). The Decay case is fully described in section~\ref{sec:DecayDB}.\\

In case of irradiation, it may also contains the evolution of the one group cross section, the evolution of the neutron flux and the keff and are not mandatory. Note that neutron flux and keff are not used in CLASS. The EvolutionData MUST contain the power and can contain the heavy metal mass, the fuel type, reactor type and the cycle time.\\
These EvolutionData can be loaded into CLASS from a formatted ASCII file see section~\ref{sec:EDformat} as follow :

\begin{center}
\begin{minipage}{\textwidth}
\begin{lstlisting}
CLASSLogger *Logger = new CLASSLogger("CLASS_OUTPUT.log",1,2);  

EvolutionData*  MyEvolutionData = new EvolutionData(Logger, "/PATH/Data.dat");
\end{lstlisting}
\end{minipage}
\end{center}


\subsection{EvolutionData ASCII format}\label{sec:EDformat}
The formatted ASCII file describing the EvolutionData is formatted as follow:
\begin{center}
\begin{minipage}{\textwidth}
\begin{lstlisting}[label=lst:DatFormat,caption=Evolution Data format]
time "0 t2 t3 ..."						// in seconds
keff "k1 k2 k3 ..."					// not mandatory entry
flux "phi1 phi2 phi3 ..." 			//(neutron/(second.cm2))not mandatory entry
Inv "Z A I inv1 inv2 inv 3 ..."		//in atoms
...
XSFis "Z A I xsfis1 xsfis2 xsfis3 ..."//in barns
... 
XSCap "Z A I xscap1 xscap2 xscap3 ..."
...
XSn2n "Z A I xsn2n1 xsnsn2 xsn2n3 ..."
...
\end{lstlisting}
\end{minipage}
\end{center}

The meaning of each keyword is listed in table~\ref{tab:meanKeyWord}.

\begin{table}[H]
\begin{center}
\caption{.dat Key words meaning}
\label{tab:meanKeyWord}
\begin{tabular}{|c|c|}
\hline
Key words & Meaning \\
\hline
Inv & Inventory \\
\hline
XSFis &   fission cross section \\
XSCap & $(n,\gamma)$ cross section\\
XSn2n  &  $(n,2n)$ cross section \\
\hline
\hline
Value & meaning \\
\hline
Z & Charge number\\
A & Mass number\\
I & State (fundamental=0, $1^{st}$ excited =1, ...) \\
\hline
\end{tabular}
\end{center}

\end{table}

Each EvolutionName.dat files comes with a EvolutionName.info file, which describes the reactor, it is formatted like this :

\begin{center}
\begin{minipage}{\textwidth}
\begin{lstlisting}
Reactor "ReactorName"		//What ever string without space
Fueltype "FuelName"			//What ever string without space
CycleTime "t"						//The final time simulated (@@BaM)
ConstantPower "P"				//Simulated power (in W)
\end{lstlisting}
\end{minipage}
\end{center}


\subsection{DecayDataBank}\label{sec:DecayDB}
The radioactive decay is handled by a DecayDataBank. The DecayDataBank contains an EvolutionData for each nucleus of the nuclei chart. 
Each EvolutionData describes the evolution of the nucleus and all its daughters as a function of the time. The depletion of an isotopic vector corresponds to the sum of all its nucleus depletion contribution.\\

In other words, in CLASS, for each nucleus of the chart, a depletion calculation has been performed and compiled in a DecayDataBank.\\
The determination  of an IsotopicVector depletion is performed as follow :\\
First, one determines the depletion of each nucleus of the IsotopicVector following the DecayDataBank, then sums all those contributions.

DecayDataBank can be defined as follow :

\begin{center}
\begin{minipage}{\textwidth}
\begin{lstlisting}
CLASSLogger *Logger = new CLASSLogger("CLASS_OUTPUT.log",1,2);
Scenario *gCLASS = new Scenario(Logger);

DecayDataBank* DecayDB = new DecayDataBank(Logger, "/PATH/Decay.idx");
gCLASS->SetDecayDataBase(DecayDB);
\end{lstlisting}
\end{minipage}
\end{center}
In the previous example a DecayDataBank has been defined using the file Decay.idx file. This file lists all the path to EvolutionDatas (each one corresponding to the depletion of one nucleus). The format of the .idx file is the following :

\begin{center}
\begin{minipage}{\textwidth}
\begin{lstlisting}
Z1 A1 I1 PATH/ZAI1.dat
...
Zn An In PATH/ZAIn.dat
\end{lstlisting}
\end{minipage}
\end{center}
A DecayDataBank can be find in \$PATH\_TO\_CLASS/Data/@@@.
\vspace{3cm}
%%%%%%%%%%%%%%%%%%%