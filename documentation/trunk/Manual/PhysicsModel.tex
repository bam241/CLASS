% !TEX root = USEGUIDE.tex

\part{Physics Models\label{sec:PhysicsModels}}
\chapter{Description and implementation}
A PhysicsModels is related to one or several reactors, it is a container of three models :
\begin{itemize}
\item Equivalence Model : Tells to the Fabrication Plant how to build the fuel.
\item XS Model : "Calculates" the mean  cross sections of this fuel and sends it to the Bateman Solver.
\item Irradiation Model : It is the Bateman Solver. User can choose between different numerical methods.
\end{itemize}

A PhysicsModels is called in the CLASS input like the following example :

\begin{center}
\begin{minipage}{\textwidth}
\textbf{Implementation in a .cxx :}
\begin{lstlisting}[style=customc,label=lst:PIMP_HYMOD, caption=PhysicsModels]
...
#include "XS/XSM_MLP.hxx"
#include "Irradiation/IM_RK4.hxx"
#include "Equivalence/EQM_PWR_MLP_MOX.hxx"
int main()
{
	....

	EQM_PWR_MLP_MOX* Equivalence 	= new EQM_PWR_MLP_MOX( "PathToTMVAWeightFile/TMVAWeightFile.xml" );
	XSM_MLP* XS = new XSM_MLP( gCLASS->GetLog(),"PathToTMVAWeighstFolder" , OneMLPPerTimeStep );
	IM_RK4* Solver = new IM_RK4( gCLASS->GetLog() );
	PhysicsModels* PHYMOD = new PhysicsModels( XS , Equivalence , Solver );
	
	...
	Reactor *PWR_MOX = new Reactor(log, PHYMOD, fabricationplant, Pool, creationtime, lifetime, cycletime, HMMass, BurnUp);
	...
}			
\end{lstlisting}
\end{minipage}
\end{center}

In this latter example a PhysicsModels called "PHYMOD" is defined, it contains the bateman solver "Solver" which is the Runge Kutta ($4^{th}$ order) method. The mean cross sections predictor, "XS",  used is  based on a Multi Layer Perceptron. The Equivalence Model "Equivalence" is the one used for PWR MOX fuels. The arguments of the 3 objects constructor are explained in their corresponding sections.

All the existing models are defined in the following sections, furthermore, the way to build a new model is presented.




%%%%%%%%%%%%%%%%%%%%%%%%%
%%		EQUIVALENCE MODEL
%%%%%%%%%%%%%%%%%%%%%%%%%
\chapter{Equivalence Model}\label{sec:EquivalenceModel}
The aim of an equivalence model is to predict the content of fissile element needed in a fuel to reach a given burnup or to satisfied criticality conditions.
\section{Available Equivalence Models}
The CLASS package contains, at the moment, 9 different equivalence models where three are related to the building of fuels for a PWR-MOX , one to the building of  PWR-UOX fuels, one for the FBR-Na MOX, two dedicated to fast breeder and one suitable for all non-breeder reactors, an other model allows to handle (Pu,Am,U)O$_{2}$ fuel loaded in PWR :
\subsection{PWR-MOX models :}
The following models returns the molar fraction $\%_{Pu}$ of plutonium needed to reach a given burnup according to the plutonium isotopic composition available in stocks.



\subsubsection{Linear BU model : EQM\_PWR\_LIN\_MOX } 
It was initially applied for MOX fuel, but because of the lack of precision, this model could be deprecated (at least in the PWR MOX case). It remain in the CLASS packages only because it was present historically.\\
 Nevertheless it could be use as an example for similar model for other fuel. This model suppose it is possible to describe the maximal burnup accessible for a set fuel using its initial composition using a simple linear modelisation (equation~\ref{eq:EQM_LIN}):\\ 
\begin{equation}\label{eq:EQM_LIN}
BU_{max} = \alpha_{0} +   \sum_{i}^{N} \alpha_{i} \cdot n_{i},
\end{equation}
where $BU_{max}$ represent the maximal accessible burnup for the fuel, $n_{i}$ the isotopic fraction of the isotope $i$, $N$ the number of isotope present in the fuel, and the $\alpha_{i}$ the parameter of the model. 
The main difficulty concerning this model, is the determination of the $\alpha_{i}$: to be correct the $\alpha_{i}$ should be fitted on a set of evolution data which are not constrain to reach an unique burnup, but a large burnup region. One can see the problem guessing it is possible to build a set a fuel evolution reaching exactly a unique burnup (45 GWd/t by example), the $\chi^{2}$ minimization of the $\alpha_i$ will end up with $\alpha_{0} = 45$ and all the other at zero. That why, when using a linear burnup description model, one should test the validity of the model, on many random compositions by example... 

\subsubsection{Quadratic Model : EQM\_PWR\_QUAD\_MOX}
The $\%_{Pu}$ is calculated according a quadratic model. See equation~\ref{eq:EQM_QUAD_MOX}.
\begin{equation}\label{eq:EQM_QUAD_MOX}
\%_{Pu} = \alpha_{0} +   \sum_{i\in Pu}^{N} \left(\alpha_{i} \cdot n_{i}\ + \sum_{j\leq i} \alpha_{ij} \cdot n_{i}\cdot n_{j}\right),
\end{equation}
where $n_{i}$ is the molar proportion (in $\%mol.$) of isotope $i$ \footnote{from $^{238}Pu$ to $^{242}Pu$} in the fresh plutonium vector.  $\alpha_{ij}$, $\alpha_{i}$ and $\alpha_{0}$  are the weights resulting from a minimization procedure and are related to one targeted burnup and one fuel management. Furthermore, $^{241}Am$ from $^{241}Pu$ decay is not one of the considered component of the model ($n_{i}$), instead the model considers a fixed time since plutonium separation. %For instance the $\alpha$ given in file \$CLASS\_PATH/DATA\_BASES/PWR/MOX/EQModel/@@BAMare representative of a PWR-MOX with a maximal burnup of $45GWd/tHM$, a fuel management of 3 batches, and a time between separation and irradiation of 2 years.

The file containing the weights is formatted as follow :

\begin{center}
\begin{minipage}{\textwidth}
\begin{lstlisting}[style=customc]
PARAM "238Pu 238Pu*238Pu 238Pu*239Pu 238Pu*240Pu 238Pu*241Pu 238Pu*242Pu 239Pu 239Pu*239Pu 239Pu*240Pu 239Pu*241Pu 239Pu*242Pu 240Pu 240Pu*240Pu 240Pu*241Pu 240Pu*242Pu 241Pu 241Pu*241Pu 241Pu*242Pu 242Pu 242Pu*242Pu 1"
\end{lstlisting}
\end{minipage}
\end{center}

Where 238Pu stands for $\alpha_{^{238}Pu}$ and it is the first order weight  related to the  molar proportion of $^{238}Pu$ and $1$ is $\alpha_{0}$. The weights are in units of $\%mol. \cdot \%mol.^{-1}$ for $\alpha_{i}$ in units of  $\%mol. \cdot \%mol.^{-2}$ for  $\alpha_{ij}$ and in units of $\%mol.$ for $\alpha_{0}$. The Keyword "PARAM" has to be present in the file before the $\alpha$ values. For more informations about this model and the generation of the coefficients please refer to reference\cite{AnnalBAM}.
\\  
\\

\begin{center}
\begin{minipage}{\textwidth}
\textbf{Implementation in a .cxx }
\begin{lstlisting}[style=customc,label=lst:IMP_EQMQUAD,caption=Equivalence Model EQM\_QUAD\_MOX ]
...
#include "Equivalence/EQM_PWR_QUAD_MOX.hxx"
...
int main()
{
...
EQM_PWR_QUAD_MOX* Equivalence = new	EQM_PWR_QUAD_MOX( LogObject, AlphasFile );
// or
// EQM_PWR_QUAD_MOX* Equivalence = new	EQM_PWR_QUAD_MOX( AlphasFile );
...
}
\end{lstlisting}
\end{minipage}
\end{center}

With LogObject a \hyperref[sec:CLASSLogger]{CLASSLogger} object (see section~\ref{sec:CLASSLogger}) and AlphasFile a string which is the complete path to the file containing the weights (the $\alpha$ parameters)
\\
\\

%\begin{center}
%\begin{minipage}{\textwidth}
%\textbf{Available weight file (.dat) : }
%\begin{itemize}
%\item \textbf{@@@ BAM}
%\item \textbf{@@@ BAM}
%\item ...
%\end{itemize}
%\end{minipage}
%\end{center}

\subsubsection{Neural network model : EQM\_PWR\_MLP\_MOX}\label{sec:EQMMOX}
This equivalence model  is based on a Multi Layer Perceptron (MLP) and  predict the amount of plutonium needed to reach \textbf{any burnup}. The MLP inputs are the isotopic compositions of the plutonium (\textbf{including $^{241}Am$}), the enrichment of depleted uranium, and the targeted burnup. The output is the plutonium content needed to reach the burnup. This method uses the neural networks of the root module TMVA \cite{TMVA}. To executes this model, TMVA is run in CLASS and need a .xml file. This file contains the neural network architecture and the weights resulting from the training procedure.
\\
\\
\begin{center}
\begin{minipage}{\textwidth}
\textbf{Implementation in a .cxx : }
\begin{lstlisting}[style=customc,label=lst:IMP_EQMMLP,caption=Equivalence Model EQM\_MLP\_PWR\_MOX ]
...
#include "Equivalence/EQM_PWR_MLP_MOX.hxx"
...
int main()
{
...
EQM_PWR_MLP_MOX* Equivalence = new	EQM_PWR_MLP_MOX( LogObject, TMVAWeightPath );
// or
// EQM_PWR_MLP_MOX.* Equivalence = new	EQM_PWR_MLP_MOX( TMVAWeightPath );
...
\end{lstlisting}
\end{minipage}
\end{center}

With LogObject a \hyperref[sec:CLASSLogger]{CLASSLogger} object (see section~\ref{sec:CLASSLogger}) and TMVAWeightPath a string containing the path to the .xml file. 


In order to make his own .xml file one need to have a training data containing the fresh fuel composition and the achievable burnup of many examples. The fuel composition is characterized by the mean of :
\begin{itemize}
\item The plutonium composition (\emph{i.e :} \%mol. of $^{238}Pu$, $^{239}Pu$, $^{240}Pu$, $^{241}Pu$, $^{242}Pu$, and $^{241}Am$)
\item The plutonium content (\emph{i.e :}  $\frac{Pu}{Pu+U}$)
\item The $^{235}U$ content in the depleted uranium.
\end{itemize}

The file \$CLASS\_PATH/DATA\_BASES/PWR/MOX/EQModel/EQM\_MLP\_PWR\_MOX\_3batch.xml has been generated from the file \$CLASS\_PATH/Utils/EQM/PWR\_MOX\_MLP/Train\_MLP.cxx 
To train a new MLP from your own training sample proceed as follow : 
\begin{center}
\begin{minipage}{\textwidth} 
\begin{lstlisting}[style=terminal]
cd $CLASS_PATH/Utils/EQM/PWR_MOX_MLP
g++ -o Train_MLP `root-config --cflags` Train_MLP.cxx `root-config --glibs` -lTMVA -I$ROOTSYS/tmva/test/
Train_MLP YourTrainingData.root
\end{lstlisting}
\end{minipage}
\end{center}


Where YourTrainingData.root is a root file containing a TTree filled with fuel compositions and corresponding burnups. The .xml file will be generated in a folder named weight. The results of the testing procedure of the MLP are in a file named TMVA\_MOX\_Equivalence.root but will be presented to you graphically as soon as the training and the testing procedure are finished.

To make your YourTrainingData.root file you have to fill a TTree with your data. To do so, create a .cxx file and copy past this :

\begin{center}
\begin{minipage}{\textwidth}
\begin{lstlisting}[style=customc]
	TFile*   fOutFile = new TFile("YourTrainingData.root","RECREATE"); //create the .root file
	TTree*   fOutT = new TTree("Data", "Data");//create the TTree
	
//---------------------INITIALISATIONN---------------------
//WARNING : keep the same variable names :
	double U5_enrichment  = 0;
	double Pu8= 0;
	double Pu9 = 0;
	double Pu10= 0;
	double Pu11= 0;
	double Pu12 = 0;
	double Am1= 0;
	double BU= 0; //BU means burnup
	double teneur              = 0; //French for content (here Pu content)
//---------------------BRANCHING--------------------------
	fOutT->Branch(	"U5_enrichment", &U5_enrichment, "U5_enrichment/D"	);
	fOutT->Branch(	"Pu8", &Pu8, "Pu8/D");
	fOutT->Branch(	"Pu9", &Pu9, "Pu9/D");
	fOutT->Branch(	"Pu10", &Pu10, "Pu10/D");
	fOutT->Branch(	"Pu11", &Pu11, "Pu11/D"	);
	fOutT->Branch(	"Pu12", &Pu12, "Pu12/D");
	fOutT->Branch(	"Am1", &Am1	, "Am1/D");
	fOutT->Branch(	"BU", &BU	,"BU/D");
	fOutT->Branch(	"teneur"	,&teneur,"teneur/D");
//---------------------FILLING--------------------------
	int Nex=NumberOfDifferentExample;
	for(int ex=0;ex<Nex;ex++)
	{ //--------Fresh Fuel Composition------------
		U5_enrichment 	=  fU5_enrichment[ex]; 
		Pu8=fPu8[ex];
		Pu9= fPu9[ex];
		Pu10=fPu10[ex];
		Pu11=fPu11[ex];
		Pu12=fPu12[ex];
		Am1=fAm1[ex];
		teneur=fteneur[ex];
		//-------Corresponding maximal burnup-------
		BU =  BurnUps[ex]; 
		//---Fill the tree with this fuel composition and this burnup------
		fOutT->Fill();
	}
	fOutFile->Write();
	delete fOutT;
	fOutFile-> Close();
	delete fOutFile;
}
\end{lstlisting}
\end{minipage}
\end{center}

Then, build the arrays fU5\_enrichment, fPu8 ... with your data, compile and execute.
For more informations about this model please refer to \cite{MLP_MOX}\\
\textbf{Available weight file (.xml) : }
\begin{itemize}
\item \textbf{\$CLASS\_PATH/DATA\_BASES/PWR/MOX/EQModel/EQM\_MLP\_PWR\_MOX\_3batch.xml} : Generated with 5000 MURE evolutions with different fuel composition, using a full mirrored assembly calculation with JEFF3.1.1 cross section and fission yield data bases. Valid for mono-recycling of plutonium and a fuel management of 3 batches. More details about the generation of this .xml file can be found in reference \cite{MLP_MOX}.
\end{itemize}

\subsection{PWR-Am model}
This model is based on the same philosophy of the $EQM\_PWR\_MLP\_MOX$ model. The only difference is in the number of inputs of the MLP (additional isotopes : Americium 241 , isomeric 242 , 243). The MLP weights given with the package are for a third batch reloading pattern. The weight are suitable to work with plutonium and americium coming from the reprocessing of PWR-UOX fuels. 

\subsection{PWR-UOX model :}

\subsubsection{Linear Model: EQM\_LIN\_UOX}
Predict the quantity of $^{235}$U needed to reach the wanted burn-up :
 \begin{equation}
N_{^{235}U} = A*BurnUp^{2} + B*BurnUp + C
 \end{equation}
See in \textbf{\$CLASS\_PATH/DATA\_BASES/PWR/UOX} for available model.

\subsection{FBR-Na-MOX model :}
This model is used to compute the plutonium content needed for a fast reactor loaded with MOX fuel.
\subsubsection{Baker \& Ross Model: EQM\_FBR\_BakerRoss\_MOX}
It calculates the plutonium content (E) needed for the FBR Na loaded with a given Pu vector according to :
 \begin{equation}
 E = \frac{E_{ref} - \sum_{fertile}N_{i}W_{i} }{\sum_{fissile}N_{i}W_{i}-\sum_{fertile}N_{i}W_{i}}
 \end{equation}
 with :
  \begin{equation}
W_i = \frac{\alpha_{i} - \alpha_{^{238}U} }{\alpha_{^{239}Pu}-\alpha_{^{238}U}}
 \end{equation}
 and
  \begin{equation}
\alpha_{i} = \bar{\nu_{i}}\cdot\sigma_{i}^{fis} - \sigma_{i}^{cap}
 \end{equation}

With $E_{ref}$ the plutonium content needed for a FBR Na to satisfy criticality condition at begining of cycle ($k_{eff}(t=0) =1.00$) with a reference fresh fuel composition. 
The reference plutonium composition is 100\% $^{239}$Pu and uranium is 100\%
 $^{238}U$. $\bar{\nu_{i}}$ is the average number of total neutron emitted per fission, $\sigma_{i}^{fis}$ is the mean fission cross section of nucleus $i$ and $\sigma_{i}^{cap}$ is the mean capture cross section of nucleus $i$. The default values of the weight $W_{i}$ given in the constructor have been calculated from a MURE/MCNP run of an ESFR lire reactor loaded with a fresh fuel composition given in table and allowing to access $k_{eff}(t=0) = 1.00$ .
To implement this model in your CLASS input proceed as follow :

\begin{center}
\begin{minipage}{\textwidth}
\textbf{Implementation in a .cxx }
\begin{lstlisting}[style=customc,label=lst:IMP_EQMBER,caption=Equivalence Model EQM\_BakerRoss\_FBR\_MOX ]
...
#include "Equivalence/EQM_FBR_BakerRoss_MOX.hxx"
...
int main()
{
...
EQM_FBR_BakerRoss_MOX* Equivalence = new	EQM_FBR_BakerRoss_MOX( ); //the default weight and Eref are used
// or
 EQM_FBR_BakerRoss_MOX* Equivalence = new EQM_FBR_BakerRoss_MOX( Weight_U_235,  Weight_Pu_238,  Weight_Pu_240, Weight_Pu_241, Weight_Pu_242, Weight_Am_241, Eref); 
;
...
}
\end{lstlisting}
\end{minipage}
\end{center}
\subsection{General non breeder model}\label{sec:GenNoBreed}
This model called EQM\_MLP\_kinf can be applied for any non breeder reactors and for fuel constituted with a fertile and a fissile part. It determines the fissile content needed to reach an user defined maximal burnup ($BU_{target}$) according a user defined number of batches $N$ (for the loading pattern) and a threshold on the multiplication factor ($k_{threshold}$).
The fissile content is varied until the maximal burnup ($BU_{max}$) is equal to $BU_{target}$.
The maximal burnup $BU_{max}$ for a given fresh fuel composition, a given number of batch and a given $k_{threshold}$ is such as : 
\[
<k_{\infty}>^{batch}(BU_{max}) = \frac{1}{N}\sum_{i=1}^{i=N-1} k_{\infty}(i*BU_{max}/N) = k_{threshold}
\]
The $k_{\infty}$ is predicted with a multi layer percetron.
To implement this model in your CLASS input proceed as follow :

\begin{center}
\begin{minipage}{\textwidth}
\textbf{Implementation in a .cxx }
\begin{lstlisting}[style=customc,label=lst:IMP_EQMBER,caption=Equivalence Model EQM\_BakerRoss\_FBR\_MOX ]
...
#include "Equivalence/EQM_MLP_Kinf.hxx"
...
int main()
{
...

 EQM_MLP_Kinf* Equivalence = new EQM_MLP_Kinf( TMVAWeightPath, NumOfBatch, InformationFile , CriticalityThreshold 
...
}
\end{lstlisting}
\end{minipage}
\end{center}
Where TMVAWeightPath is the path to the weight file of the MLP (.xml file) , NumOfBatch is the number of batches for the loading pattern and CriticalityThreshold is the $k_{threshold}$. InformationFile contains information regarding the MLP inputs and are listed above (the quotes have to be removed):

\begin{center}
\begin{minipage}{\textwidth}
\begin{lstlisting}[style=customc,label=lst:infoEQMKinf,caption=Information file format]
Specific Power (W/gHM) :
k_specpower "specificpower" // the power density in Watt per gram of heavy metal

Maximal burnup (GWd/tHM) : // for the algorithm initialization : a relatively high Burnup value
k_maxburnup "BUmax"		  // e.g 100 for a PWR 	

Maximal fissile content (molar proportion) : // for the algorithm initialization : a relatively high fissile content
k_maxfiscontent "maxFisContent"  // e.g 0.25 for a PWR MOX

Z A I Name (input MLP) : // name for the MLP inputs
k_zainame "Z1 A1 I1 Name1"
...
k_zainame "Z2 A2 I2 Name2"
...
Fissile Liste (Z A I) : // the fissile list to be taken in the stocks for fuel manufacturing
k_fissil "Z1 A1 I1"
..
k_fissil "Z2 A2 I2"

Fertile Liste (Z A I Default Proportion) :// the fertile list to be taken in the stocks for fuel manufacturing
k_fertil "Z1 A1 I1 prop"
..
k_fertil "Z2 A2 I2 prop2"
\end{lstlisting}
\end{minipage}
\end{center}

A weight file (.xml) and .nfo file can be found in : \\
\$CLASS\_PATH/DATA\_BASES/PWR/MOX/EQModel/MLP\_Kinf/MLP

In order to make his own .xml file one need to have a training data containing the fresh fuel composition and the k evolution over time.\\
First, you have to convert your depletion calculations in .dat format (see section~\ref{sec:EDformat} for the fomat definition). If you use MURE depletion code, you can find the convertor software in : \\ 
\$CLASS\_PATH/Utils/MURE2CLASS/.\\
Once the set of .dat files is generated you can convert this files in one .root file to be red by the training algorithm of the MLP. The software is located in :\\
\$CLASS\_PATH/Utils/EQM/MLP\_Kinf/GenerateRootFile.cxx. 
Go to this folder and edit this .cxx file and look for @@@change to make the apropriate changes. Then compile and execute.\\
Then, you are good to go to the training/testing process : 
 Edit TrainMLP.cxx and make the appropriate changes (looking for @@@changes).Then compile and execute. You should find your .xml file in : \\
 \$CLASS\_PATH/Utils/EQM/MLP\_Kinf/weight/ .
 You can also check the testing results\\
Create a folder in \$CLASS\_PATH/DATA\_BASES with the name you want. Move the .xml and .nfo in this location , make sure these two files have the same name (except their extension of course), and voila. 



\subsection{General breeder models}
\subsubsection{ $k_{eff}(t=0)$ prediction using MLP }
This model aims to predict the fissile content satisfying $k_{eff}(t=t_{user})=k_{user}$. A MLP is used to predict the $k_{eff}$ for a given irradiation time. Then the fissile content is adjusted until $k_{eff}=k_{user}$. A MLP weight file is given in \$DATA\_BASES/FBR\_Na/MOX/EQModel/MLP\_K\_EFF\_BOC and is tuned to predict the $k_{eff}$ of an ESFR like reactor loaded with MOX fuel at BOC ($t_{user} = 0$). To change the $t_{user}$ you have to train your neural network to predict $k_{eff}$ at this irradiation time.\\
In order to make his own .xml file one need to have a training data containing the fresh fuel composition and the k at the $t_{user}$.\\
First, you have to convert your depletion calculations in .dat format (see section~\ref{sec:EDformat} for the fomat definition). If you use MURE depletion code, you can find the convertor software in : \\ 
\$CLASS\_PATH/Utils/MURE2CLASS/.\\
Once the set of .dat files is generated you can convert this files in one .root file to be red by the training algorithm of the MLP. The software is located in :\\
\$CLASS\_PATH/Utils/EQM/FBR\_MLP\_Keff/GenerateRootFile.cxx. 
Go to this folder and edit this .cxx file and look for @@@change to make the apropriate changes. Then compile and execute.\\
Then, you are good to go to the training/testing process : 
 Edit TrainMLP.cxx and make the appropriate changes (looking for @@@changes).Then compile and execute. You should find your .xml file in : \\
 \$CLASS\_PATH/Utils/EQM/FBR\_MLP\_Keff/weight/ .
 You can also check the testing results\\
Create a folder in \$CLASS\_PATH/DATA\_BASES with the name you want. Move the .xml and .nfo in this location , make sure these two files have the same name (except their extension of course), and voila. 

For this model to work a nfo file is also required the format is given above :

\begin{center}
\begin{minipage}{\textwidth}
\begin{lstlisting}[style=customc,label=lst:infoEQMKeffFBR,caption=Information file format]
Specific Power (W/gHM) :
k_specpower "specificpower" // the power density in Watt per gram of heavy metal

Maximal fissile content (molar proportion) : // for the algorithm initialization : a relatively high fissile content
k_maxfiscontent "maxFisContent"  // e.g 0.4 for a FBR MOX

Z A I Name (input MLP) : // name for the MLP inputs
k_zainame "Z1 A1 I1 Name1"
...
k_zainame "Z2 A2 I2 Name2"
...
Fissile Liste (Z A I) : // the fissile list to be taken in the stocks for fuel manufacturing
k_fissil "Z1 A1 I1"
..
k_fissil "Z2 A2 I2"

Fertile Liste (Z A I Default Proportion) :// the fertile list to be taken in the stocks for fuel manufacturing
k_fertil "Z1 A1 I1 prop"
..
k_fertil "Z2 A2 I2 prop2"
\end{lstlisting}
\end{minipage}
\end{center}

To implement this model in your CLASS input proceed as follow :

\begin{center}
\begin{minipage}{\textwidth}
\textbf{Implementation in a .cxx }
\begin{lstlisting}[style=customc,label=lst:IMP_EQMBER,caption=Equivalence Model EQM\_BakerRoss\_FBR\_MOX ]
...
#include "Equivalence/EQM_FBR_MLP_Keff.hxx"
...
int main()
{
...

EQM_FBR_MLP_Keff* Equivalence = EQM_FBR_MLP_Keff( TMVAWeightPath,  keff_user )
//TMVAWeightPath is the path to the .xml file
..
}
\end{lstlisting}
\end{minipage}
\end{center}

\subsubsection{ Upper and lower limits on $<k_{\infty}>^{batch}$}
In order to take into account the impact of the loading pattern this model used the $<k_{\infty}>^{batch}$ function defined in section~\ref{sec:GenNoBreed}. The fissile content is such as this value is contained in a user defined range. We suggest to use $1/P_{noleak}$ as the lower bound and $1/P_{noleak} + |\rho_{controlRods}|$ as upper bound. With $P_{noleak}$ is the no leak probability ($\sim$ 0.88 for ESFR like reactor) and $\rho_{controlRods}$ the anti-reactivity of the control rods (estimated for a ESFR like to be 2000 pcm for all control rods put at half of the core high). Not than in rare occasion no solution are available for a given fissile and fertile composition. The $k_{\infty}(t)$ is determined using a MLP.
In order to make his own .xml follow the same procedure as the one explained in section~\ref{sec:GenNoBreed}.


To implement this model in your CLASS input proceed as follow :

\begin{center}
\begin{minipage}{\textwidth}
\textbf{Implementation in a .cxx }
\begin{lstlisting}[style=customc, label=lst:IMP_EQMBER,caption=Equivalence Model EQM\_BakerRoss\_FBR\_MOX ]
...
#include "Equivalence/EQM_FBR_MLP_Kinf_BOUND.hxx"
...
int main()
{
...

EQM_FBR_MLP_Kinf_BOUND* Equivalence = EQM_FBR_MLP_Kinf_BOUND( TMVAWeightPath, NumOfBatch ,  LowerK,  UpperK)
//TMVAWeightPath is the path to the .xml file
// NumOfBatch is the number of batches for the fuel loading pattern
// LowerK is the lower bound on <k_{\infty}>^{batch}$
// UpperK is the upper bound on <k_{\infty}>^{batch}$

..
}
\end{lstlisting}
\end{minipage}
\end{center}
For this model to work a nfo file is also required the format is given above :

\begin{center}
\begin{minipage}{\textwidth}
\begin{lstlisting}[style=customc,label=lst:infoEQMKeffFBR,caption=Information file format]
Specific Power (W/gHM) :
k_specpower "theSpecificPower" // i.e the power density in watt per gram of heavy metal

Time (s) :// the time used to train the MLP
K_TIMESTEP "0 t1 t2 ..."


Z A I Name (input MLP) : // name for the MLP inputs
k_zainame "Z1 A1 I1 Name1"
...
k_zainame "Z2 A2 I2 Name2"
...
Fissile Liste (Z A I) : // the fissile list to be taken in the stocks for fuel manufacturing
k_fissil "Z1 A1 I1"
..
k_fissil "Z2 A2 I2"

Fertile Liste (Z A I Default Proportion) :// the fertile list to be taken in the stocks for fuel manufacturing
k_fertil "Z1 A1 I1 prop"
..
k_fertil "Z2 A2 I2 prop2"
\end{lstlisting}
\end{minipage}
\end{center}

A weight file (.xml) and .nfo file can be found in \\
\$CLASS\_PATH/DATA\_BASES/FBR\_Na/MOX/EQModel/MLP\_K\_INF\_BOUND

\section{How to build an Equivalence Model}
The strength of CLASS is to allow the user to build his own Physics models, this section explains how to build a new equivalence model and to incorporate it into CLASS.

First you have to create the file EQM\_NAME.cxx and EQM\_NAME.hxx, where NAME is a name you choose. 
Then open with a text editor the .hxx and copy past the following replacing NAME by the name you want.
\begin{center}
\begin{minipage}{\textwidth}
\begin{lstlisting}[style=customc,label=lst:HXX_EQM,caption=EQM\_NAME.hxx ]
#ifndef _EQM_NAME_HXX
#define _EQM_NAME_HXX
#include "EquivalenceModel.hxx"
using namespace std;
/*! Define a EQM_NAME
  Explain briefly what is it.
  @author YourName
  @version 3.0 
*/

class EQM_NAME : public EquivalenceModel
{
	public :
	/*Constructor*/
	EQM_NAME(/*parameters*/ ); //!< Explain what is the parameters (if any) 

	/**This function IS the equivalence model **/	
	double GetFissileMolarFraction(IsotopicVector Fissil,IsotopicVector Fertil,double BurnUp); //!<Return the molar fraction of fissile element 

	private :
	/*Your private variables*/
};
#endif
\end{lstlisting} 
\end{minipage}
\end{center}

Open the .cxx file and copy past the following in it (replacing NAME by the same name you used in the .hxx).
 
\begin{center}
\begin{minipage}{\textwidth}
\begin{lstlisting}[style=customc,label=lst:CXX_EQM,caption=EQM\_NAME.cxx ]
#include "EquivalenceModel.hxx"
#include "EQM_NAME.hxx"
#include "CLASSLogger.hxx"
/*Whatever include you need*/
//________________________________________________________________________
//		EQM_NAME
//
//	Brief description
//________________________________________________________________________
//Constructor(s)
EQM_NAME::EQM_NAME(/*parameters*/)
{
//.... Do whatever you want with your parameters
/*
	Fill the two isotopic vectors fFissileList and fFertileList 
	see explanation in the manual
*/
	//Fertile
	ZAI U8(92,238,0);
	ZAI U5(92,235,0);
	double U5_enrich= 0.0025;
	fFertileList = U5*U5_enrich + U8*(1-U5_enrich);

	//Fissile
	ZAI Pu8(94,238,0);
	ZAI Pu9(94,239,0);
	//...
	fFissileList = Pu8*1+Pu9*1+ /*...*/;
}
//_______________________________________________________________________
double EQM_NAME::GetFissileMolarFraction(IsotopicVector Fissil,IsotopicVector Fertil,double BurnUp)
{
//Code your Equivalence Model : This function has to return the molar fraction	of fissile in the fuel needed to reach the BurnUp(GWd/tHM) according to the composition of the Fissil and Fertil vectors 
}
\end{lstlisting} 
\end{minipage}
\end{center}

In the constructor (EQM\_NAME::EQM\_NAME) you have to fill two isotopic vectors named \textbf{fFissileList} and \textbf{fFertileList}. Don't declare these isotopic vector in the .hxx, there are already declared in the file src/EquivalenceModel.hxx. fFissileList is used by the FabricationPlant to do the chemical separation of the fissile element from the other present in stock. For instance, for the plutonium, add the ZAI $^{238}Pu$, $^{239}Pu$, $^{240}Pu$, $^{241}Pu$ and $^{242}Pu$. fFertile List is used by the FabricationPlant the same way fFissileList is used but you have to define a default \hyperref[sec:IsotopicVector]{IsotopicVector} to be used if you didn't provide a fertile stock to your FabricationPlant.  In the example given above the fertile is depleted uranium and  the proportion of each isotope is given ($^{234}U$ is unheeded). Now you have to build the function \textbf{GetFissileMolarFraction(IsotopicVector Fissil, IsotopicVector Fertil, double BurnUp)}. Its parameters are provided by the FabricationPlant and are :

\begin{itemize}
\item \hyperref[sec:IsotopicVector]{IsotopicVector} Fissile : it is the proportion of each nucleus you give in the fFissileList plus the proportion of the nuclei that appears during the fabrication time (time given in the FabricationPlant constructor, is default is 2 years)
\item \hyperref[sec:IsotopicVector]{IsotopicVector} Fertile :  it is the proportion of each nucleus you give in the fFertileList plus the proportion of the nuclei that appears during the fabrication time. If you didn't provide any fertile stock to your FabricationPlant then it's the default vector given in the EQM\_NAME constructor.
\item double BurnUp : The maximal average burnup for your fuel to reach (in GWd/tHM).
\end{itemize}
Fill free to have a look at the models present in \$CLASS\_PATH/source/Model/Equivalence to get inspiration.

Now that your equivalence model is ready two choices are offered to you. You can compile the two files of your model with your CLASS input or you can add this model to the CLASS package. The second option will modify the CLASS software and we will be no longer able to troubleshoot your scenario. So use the second option only if you are a completely independent user !

\subsection{Compile your equivalence model with your CLASS executable :}
\begin{center}
\begin{minipage}{\textwidth}
\begin{lstlisting}[style=terminal,label=lst:Compile]
g++ -g -O -I $CLASS_include -L $CLASS_lib -lCLASSpkg `root-config --cflags` 
	`root-config --libs` -fopenmp -lgomp -Wunused-result -c My_MODEL.cxx
		
\rm CLASS* ; g++ -o CLASS_exec MyScenario.cxx My_MODEL.o -I $CLASS_include -L $CLASS_lib -lCLASSpkg `root-config --cflags` `root-config --libs` -fopenmp -lgomp -Wunused-result

\end{lstlisting}
\end{minipage} 
\end{center}

\subsection{Your equivalence model in the CLASS library :}
Move your  EQM\_NAME.hxx and  EQM\_NAME.cxx in \$CLASS\_PATH/source/Model/Equivalence/. Then open with your favourite text editor the file  \$CLASS\_PATH/source/src/Makefile, find "OBJMODEL" and add \$(EQM)/EQM\_NAME.o within the others \$(EQM) objects. Then re-compile CLASS, fix the compilation errors ;) and voil� your equivalence model is now available in the CLASS library.




%%%%%%%%%%%%%%%%%%%%%%%%%
%%		XS MODEL
%%%%%%%%%%%%%%%%%%%%%%%%%
\chapter{XS Model}
The aim of a mean cross section model (XSModel) is to predict the mean cross sections of a fuel built by an EquivalenceModel (EQM) (see section \ref{sec:EquivalenceModel}). The mean cross sections are required to compute fuel depletion in a reactor.

\section{Available XS Models}
There is, for the moment, 2 XSModel in CLASS : 
\subsection{Pre-calculated XS : XSM\_CLOSEST}\label{sec:CLOSEST}
This method looks, in a data base, for a fresh fuel with a composition \textbf{close} to the brandy new fuel built by the EquivalenceModel. Here, close means that the fresh fuel in the data base minimizes the distance $d$ (see equation~\ref{eq:distance}).
\begin{equation}\label{eq:distance}
d=\sqrt{\sum_{i} w_{i}\cdot(n_{i}^{DB} - n_{i}^{new} )^{2} },
\end{equation}
where $n_{i}^{DB}$ is the number of nuclei $i$ in one element of the data base and  $n_{i}^{new}$ the number of nuclei $i$ in the new fuel built by the EQM. $w_{i}$ is a weight associated to each isotopes, its value is 1 by default.
When the closest evolution in the database is found, the corresponding mean cross sections are extracted and used for the calculation of the depletion of the new fuel.
\\
\\
\begin{center}
\begin{minipage}{\textwidth}
\textbf{Implementation in a .cxx : }
\begin{lstlisting}[style=customc,label=lst:IMP_XSMCLOSEST,caption=Cross section Model XSM\_CLOSEST ]
...
#include "XS/XSM_CLOSEST.hxx"
...
int main()
{
	XSM_CLOSEST* XSMOX = new XSM_CLOSEST( gCLASS->GetLog(), PathToIdxFile );
	//or
	//XSM_CLOSEST* XSMOX = new XSM_CLOSEST( PathToIdxFile );
}
\end{lstlisting}
\end{minipage}
\end{center}

With LogObject a \hyperref[sec:CLASSLogger]{CLASSLogger} object (see section~\ref{sec:CLASSLogger}) and PathToIdxFile  a string containing the path to the .idx file. The .idx file lists all the EvolutionData (see section~\ref{sec:EvolutionData}) of the data base. This file is formatted as follow :

\begin{center}
\begin{minipage}{\textwidth}
\begin{lstlisting}[style=customc]
TYPE "NameOfTheFuel(withoutspace)"
"PATH_TO_DATA_BASE/EvolutionName.dat"
"PATH_TO_DATA_BASE/OtherEvolutionName.dat"
....
\end{lstlisting}
\end{minipage}
\end{center}

Each EvolutionName.dat file contains a formatted fuel depletion calculation. the format of a EvolutionData ASCII file is detailed in section~\ref{sec:EDformat}.  The number of .dat files has an influence on the model accuracy. Furthermore, the initial composition of the different fuel depletion calculations has to be representative of the fresh fuel compositions encounter in a scenario. %For more details on this method please refer to \cite{PhysorBAM}.


%\textbf{Available .idx file  : }
%\begin{itemize}
%\item \textbf{@@@ BAM}
%\item \textbf{@@@ BAM}
%\item ...
%\end{itemize}

\textbf{For MURE user only : } The program \$CLASS\_PATH/Utils/MURE2CLASS converts a list of MURE evolutions to a list of .dat and .info files and creates the .idx file, type in terminal the following command for more details.

\begin{center}
\begin{minipage}{\textwidth}
\begin{lstlisting}[style=terminal]
\$CLASS\_PATH/Utils/MURE2CLASS -h 
\end{lstlisting}
\end{minipage}
\end{center}

Users of others fuel depletion code (\emph{e.g} VESTA, ORIGEN, MONTEBURNS, SERPENT .... ) have to create their own program to generate these files.




\subsection{XS predictor : XSM\_MLP}
This method calculates the mean cross sections by the mean of a set of neural networks (MLP from TMVA module) . There is two configurations available :

\begin{itemize}
\item One MLP per nuclear reaction and per time step (this one is deprecated and not describe in this manual) .
\item	One MLP per nuclear reaction. the irradiation time is one of the MLP inputs.
\end{itemize}

\begin{center}
\begin{minipage}{\textwidth}
\textbf{Implementation in a .cxx : }
\begin{lstlisting}[style=customc,label=lst:IMP_XSMMLP,caption=Cross section Model XSM\_MLP ]
...
#include "XS/XSM_MLP.hxx"
...
int main()
{	...
	XSM_MLP* XSMOX = new XSM_MLP( ClassLog, PathToWeightFolder, InfoFileName, OneMLPPerTime );
//or
//XSM_MLP* XSMOX = new XSM_MLP(PathToWeightFolder, InfoFileName, OneMLPPerTime);
...
}
\end{lstlisting}
\end{minipage}
\end{center}
\textbf{PathToWeightFolder} (string) is the path to the folder containing the weight files (.xml files). \textbf{OneMLPPerTime} is a boolean set to true if there is one MLP per reaction and per time step. \textbf{InfoFileName} (string) is the name of the file located in PathToWeightFolder which is informing on the reactor and on the inputs of the XS\_MLP model. Its default name is Data\_Base\_Info.nfo .
The InfoFileName contains keywords beginnings with k\_ (note that it is not case sensitive) and corresponding value(s). Any comments can be added. The quotes must be removed.
\begin{center}
\begin{minipage}{\textwidth}
\begin{lstlisting}[style=customc,label=lst:informationfile,caption=Information file format]
ReactorType :
K_REACTOR "ReactorName"	//without space
FuelType :
K_FUEL "FuelName" //without space
Heavy Metal (t) :
K_MASS "m"
Thermal Power (W) :
K_POWER "P"     //power corresponding to the heavy metal mass
Time (s) :
K_TIMESTEP "0 t2 t3 t4 ..." //Time when the cross section are updated
Z A I Name (input MLP) : //see explanations below
K_ZAINAME "z a i InputName"
K_ZAINAME "z2 a2 i2 InputName2"
"..."
Fuel range (Z A I min max) :
K_ZAIL "z a i min max" //minimal and maximal proportion of the zai in the fresh fuel (heavy nuclei only, ie without oxygen)"
K_ZAIL "z2 a2 i2 min2 max2"
\end{lstlisting}
\end{minipage}
\end{center}

The input of MLPs are the atomic proportion of each nuclei present in the fresh fuel (plus time if OneMLPPerTime=false). The InfoFile has to indicates the variable names (nuclei name) you used for the \textcolor{blue}{\textbf{training of your MLPs}}. For instance if the fresh fuel contains $^{238}Pu$ you will write in the InfoFile  :

\begin{center}
\begin{minipage}{\textwidth}
\begin{lstlisting}[style=customc]
...
Z A I Name (input MLP) :
K_ZAINAME 94 238 0 Pu8//(if Pu8 is the variable name used for 238Pu proportion in fresh fuel in your training sample) 
...
\end{lstlisting}
\end{minipage}
\end{center}

The tag "Fuel range (Z A I min max) :" corresponds to the validity domain of the XSM\_MLP model. This indication is not mandatory but its useful to know if the fuel we calculate the cross section is in the domain of validity of the model.

\textbf{Available XSM\_MLP : }
\begin{itemize}
\item \$CLASS\_PATH/DATA\_BASES/PWR/MOX/XSModel/30Wg\_FullMOX :
The weight files and .nfo file contained in this folder are representative of a PWR MOX. With the MOX coming from PWR UO2 spent fuels. The specific power is 30W/g oxide. To perform this data base, MURE depletion calculations have been performed using a full MOX assembly with mirror boundaries. 
\item \$CLASS\_PATH/DATA\_BASES/FBR\_Na/MOX/XSModel/ESFR\_48Wg :
The weight files and .nfo file contained in this folder are representative of a FBR-Na MOX. The specific power is 48W/g oxide. To perform this data base, MURE depletion calculations have been performed using a 1/12 of ESFR like core with mirror boundaries. 
\item \$CLASS\_PATH/DATA\_BASES/PWR/UOX/XSModel/30Wg\_FullUOX :
The weight files and .nfo file contained in this folder are representative of a PWR UOX. The specific power is 30W/g oxide. To perform this data base, MURE depletion calculations have been performed using a full UOX assembly with mirror boundaries. 
\item \$CLASS\_PATH/DATA\_BASES/PWR/MOX\_Am/XSModel/30Wg\_FullMOX\_Am :
The weight files and .nfo file contained in this folder are representative of a PWR loaded with (Pu,U,Am)$O_{2}$. Plutonium and Americium compositions are representative of compositions in UOX spent fuels. The specific power is 30W/g oxide. To perform this data base, MURE depletion calculations have been performed using an assembly with mirror boundaries. 
\end{itemize}

\textcolor{blue}{\large{\textbf{Training MLPs for cross sections prediction :}}}\\

\textbf{\underline{ Preparation of the training sample :} }\\
\\
Like for the equivalence model, first of all you have to create a training sample. This is one of the most important thing since the way of filling the hyperspace of the MLP inputs will influence the accuracy of your model. We suggest to used the Latin Hyper Cube method \cite{LHS} to generate many fresh fuel compositions, then, calculates with your favourite neutron transport code (MCNP, MORET, SERPENT ...) the mean cross sections of each fresh fuel for different irradiation time. Please refer to [REFFFBAL MLPXS] for more informations about the space filling and the validation of this cross sections predictor.
Once all your calculations are complete you have to convert them into the .dat format (see code frame~\ref{lst:DatFormat}).
Then type :

\begin{center}
\begin{minipage}{\textwidth}
\begin{lstlisting}[style=terminal]
cd $CLASS_PATH/Utils/XS/MLP/BuildInput
\end{lstlisting}
\end{minipage}
\end{center}

Open the file Gene.cxx, looks for @@Change and make the appropriate changes. Then type :

\begin{center}
\begin{minipage}{\textwidth}
\begin{lstlisting}[style=terminal]
g++ -o Gene Gene.cxx `root-config --cflags` `root-config --libs`
Gene PATH_To_dat_Folder/
\end{lstlisting}
\end{minipage}
\end{center}

Where  PATH\_To\_dat\_Folder/ is the path to the folder containing the .dat files. This program should have built two files :

\begin{itemize}
\item TrainingInput.root : This root file contains the fresh fuel inventories and the cross sections values of all the read .dat files. You can plot the data with the root command line tool if you wish. This file is the \textbf{Training and testing sample} that will be used for the TMVA training and testing procedure.
 
\item TrainingInput.cxx : This file contains, in a vector, the names of all the MLP outputs. The number of lines in this file is the number of MLP that will be train.
\end{itemize} 

\textbf{\underline{ Training and testing procedure : }}\\
\\
Once the two TrainingInput (.cxx and .root) are generated type :

\begin{center}
\begin{minipage}{\textwidth}
\begin{lstlisting}[style=terminal]
cd $CLASS_PATH/Utils/XS/MLP/Train
\end{lstlisting}
\end{minipage} 
\end{center}

 Look for @@Change in the file Train\_XS.cxx , and make the appropriate changes. Then type :

\begin{center}
\begin{minipage}{\textwidth}
\begin{lstlisting}[style=terminal]
 g++ -o Train_XS  `root-config --cflags` Train_XS.cxx `root-config --glibs` -lTMVA
\end{lstlisting}
\end{minipage} 
\end{center}

According the number of "events" in your .root file and the number of cross sections, the training time can be very very very long. You might want to decrease the number of events (this will probably deteriorate the model accuracy) : look for nTrain\_Regression in Train\_XS.cxx and change its value to your wanted number of events. And/Or you may want to use more than one processor or perhaps a supercomputer : This is completely doable since the program Train\_XS trains only one MLP (one cross section). Indeed the execution line is the following :

\begin{center}
\begin{minipage}{\textwidth}
\begin{lstlisting}[style=terminal]
Train_XS i
\end{lstlisting}
\end{minipage} 
\end{center}

where i is the index of the cross section in the vector created in TrainingInput.cxx. So feel free to create a script to run the training on a wanted number of processors. For instance let's say you have 40 cross sections and 4 processors, creates 4 files (make them executable) and in the first one type :

\begin{center}
\begin{minipage}{\textwidth}
\begin{lstlisting}[style=customc]
Train_XS 0
Train_XS 1
...
TrainXS 9
\end{lstlisting}
\end{minipage} 
\end{center}

continue in the second file, and so on. Then execute all of them. The architecture and weights of each MLP (.xml files) are stored in the folder weights. Rename this folder by the name of the reactor and fuel, then create in this folder the information file (see code frame~\ref{lst:informationfile}). And voil� your new XSM\_MLP is ready to be used.\\

After each training (using by default the half of the events) a testing procedure (using the other half) is performed. This latter consists on executing the trained MLP with input data from a known sample and compare the MLP result to the true value. These data and other informations about the training are stored in file \textbf{Training\_output\_i.root}, with i the index of the cross section. In order to see either the MLPs predictions are accurate or not, the root macro \$CLASS\_PATH/Utils/XS/MLP/Train/deviations.C plot the distribution of relative differences between model executions and the true values and a  Gaussian fit of it. Then, the mean and the standard deviation of the Gaussian fit are stored in file \textbf{XS\_accuracy.dat} (format : XSName mean std.dev.). Type the following to get, in file XS\_accuracy.dat, the mean and the standard deviation of all the MLPs (with N the number of cross sections (number of MLPs) ) :

\begin{center}
\begin{minipage}{\textwidth}
\begin{lstlisting}[style=terminal]
cd $CLASS\_PATH/Utils/XS/MLP/Train/
root
.L deviations.C
for(int i=0;i<N;i++) {stringstream ss;ss<<"Training_output_"<<i<<".root";deviations(ss.str().c_str(),0,kTRUE,kFALSE,kFALSE); }
\end{lstlisting}
\end{minipage} 
\end{center}
 
 The closest to 0 the mean is and the smaller standard deviation, the better.
 
 

\section{How to build an XS Model}
The strength of CLASS is to allow the user to build his own Physics models, this section explains how to build a new cross section model and to incorporate it into CLASS.
First you have to create the file XSM\_NAME.cxx and XSM\_NAME.hxx, where NAME is a name you choose. 
Then open with a text editor the .hxx and copy past the following replacing NAME by the name you want.

\begin{center}
\begin{minipage}{\textwidth}
\begin{lstlisting}[style=customc,label=lst:HXX_XSM,caption=XSM\_NAME.hxx ]
#ifndef _XSM_NAME_HXX
#define _XSM_NAME_HXX
#include "XSModel.hxx"
// add include if needed
using namespace std;
//----------------------------------------------------------------------//
/*!
 Define a XSM_NAME
describe your model
 @authors YourName
 @version 1.0
 */
//________________________________________________________________________
class XSM_NAME : public XSModel
{
	public :
	
	XSM_NAME(/*parameters (if any)*/);

	~XSM_NAME();

 	EvolutionData GetCrossSections(IsotopicVector IV,double t=0);

	private :			
	//your private variables and methods
};
#endif
\end{lstlisting}
\end{minipage} 
\end{center}


Open the .cxx file and copy past the following in it (replacing NAME by the same name you used in the .hxx).
 
\begin{center}
\begin{minipage}{\textwidth}
\begin{lstlisting}[style=customc,label=lst:CXX_XSM,caption=XSM\_NAME.cxx ]
#include "XSModel.hxx"
#include "XSM_NAME.hxx"
#include "CLASSLogger.hxx"
#include "StringLine.hxx"

#include <TGraph.h>
//________________________________________________________________________
//
//		XSM_NAME
//________________________________________________________________________
XSM_NAME::XSM_NAME(/*parameters (if any)*/)
{
// do what you want : for instance save path of eventual files
}
//________________________________________________________________________
XSM_NAME::~XSM_NAME()
{
 //delete pointer if any; clear map if any ; empty vector if any
}
//________________________________________________________________________
EvolutionData XSM_NAME::GetCrossSections(IsotopicVector IV ,double t)
{
	EvolutionData EvolutionDataFromXSM_NAME = EvolutionData();
	/*************DATA BASE INFO****************/
	EvolutionDataFromXSM_NAME.SetReactorType(fDataBaseRType);//Give the reactor name
	EvolutionDataFromXSM_NAME.SetFuelType(fDataBaseFType);//Give the fuel name
	EvolutionDataFromXSM_NAME.SetPower(fDataBasePower);//Set the power W
	EvolutionDataFromXSM_NAME.SetHeavyMetalMass(fDataBaseHMMass);//corresponding to this mass (t)
	
	map<ZAI,TGraph*> ExtrapolatedXS[3];
//... Fill the 3 maps ExtrapolatedXS  according to your model and the
// fresh fuel composition given by argument IsotopicVector IV 
// argument double t may be not used. 	

	/*****THE CROSS SECTIONS***/
	EvolutionDataFromXSM_NAME.SetFissionXS(ExtrapolatedXS[0]);
	EvolutionDataFromXSM_NAME.SetCaptureXS(ExtrapolatedXS[1]);
	EvolutionDataFromXSM_NAME.Setn2nXS(ExtrapolatedXS[2]);

return EvolutionDataFromXSM_NAME;
}
\end{lstlisting}
\end{minipage} 
\end{center}

Then, edit these two files to make the function XSM\_NAME::GetCrossSections to return the cross sections in a EvolutionData object. (\textit{In this case, the EvolutionData only contains the 1 group cross section without the inventory evolution, the power and the corresponding mass.})\\
 To do so you have to fill three maps (ExtrapolatedXS in .cxx), one for fission, one for $(n,\gamma)$, and one for $(n,2n)$ . Each map associates a nucleus (a ZAI) to a TGraph. A TGraph is a root object, here, it contains the cross section (barns) evolution over time (seconds). If your are not comfortable with TGraph refer to the  \href{http://root.cern.ch/root/html/TGraph.html}{root website}
\footnote{http://root.cern.ch/root/html/TGraph.html}

Now that your cross section model is ready, two choices are offered to you. You can compile the two files of your model with your CLASS input or you can add this model to the CLASS package. The second option will modify the CLASS software and we will be no longer able to troubleshoot your scenario. So use the second option only if you are a completely independent user !

\subsection{Compile your cross section model with your CLASS executable :}
\begin{center}
\begin{minipage}{\textwidth}
\begin{lstlisting}[style=terminal,label=lst:Compile]
g++ -g -O -I $CLASS_include -L $CLASS_lib -lCLASSpkg `root-config --cflags` 
	`root-config --libs` -fopenmp -lgomp -Wunused-result -c My_MODEL.cxx
	
\rm CLASS* ; g++ -o CLASS_exec MyScenario.cxx My_MODEL.o -I $CLASS_include -L $CLASS_lib -lCLASSpkg `root-config --cflags` `root-config --libs` -fopenmp -lgomp -Wunused-result
\end{lstlisting}
\end{minipage} 
\end{center}

\subsection{Your cross section model in the CLASS library :}
Move your  XSM\_NAME.hxx and  XSM\_NAME.cxx in \$CLASS\_PATH/source/Model/XS/. Then open with your favourite text editor the file \\ \$CLASS\_PATH/source/src/Makefile, find "OBJMODEL" and add \$(XSM)/XSM\_NAME.o within the others \$(XSM) objects. Then re-compile CLASS, fix the compilation errors ;) and voil� your cross section model is now available in the CLASS library.





%%%%%%%%%%%%%%%%%%%%%%%%%
%% IRRADIATION MODEL
%%%%%%%%%%%%%%%%%%%%%%%%%
\chapter{Irradiation Model}

\textbf{The irradiation model is the Bateman equations solver}. It is used for the calculation of fuel depletion in reactor. The decay depletion (without neutron flux) is not managed by an irradiation model but with a decay data bases (see section~\ref{sec:DecayDB}). 

\section{Available Irradiation Model}
At the moment, there is two Irradiation Model available. The two solvers differs according to the numerical integration method used. The Irradiation Model IM\_RK4 uses the fourth order Runge-Kutta method. And IM\_Matrix uses the development in a power series of the exponential of the Bateman matrix.

\begin{center}
\begin{minipage}{\textwidth}
\textbf{ Implementation in a .cxx :}
\begin{lstlisting}[style=customc,label=lst:IMP_IRM,caption=Irradiation Model ]
#include "CLASSHeaders.hxx"
#include "Irradiation/IM_RK4.hxx"
//#include "Irradiation/IM_Matrix.hxx"
..
using namespace std;
int main()
{
//...
	IM_RK4* Solver = new IM_RK4(LogObject); // or new IM_RK4(); // uses a default logfile
//	IM_Matrix* Solver = new IM_Matrix(LogObject); // or new IM_Matrix(); //uses default logfile
	PhysicsModels* PHYMOD = new PhysicsModels(XSMOX, EQMLINPWRMOX, Solver);
//...	
}
\end{lstlisting}
\end{minipage} 
\end{center}
LogObject is a  \hyperref[sec:CLASSLogger]{CLASSLogger} object (see section~\ref{sec:CLASSLogger}).

\subsection{How to build an Irradiation Model}
The strength of CLASS is to allow the user to build his own Physics models, this section explains how to build a new Bateman solver (Irradiation Model) and to incorporate it into CLASS.
First you have to create the file IRM\_NAME.cxx and IRM\_NAME.hxx, where NAME is a name you choose. 
Then open with a text editor the .hxx and copy past the following replacing NAME by the name you want.

\begin{center}
\begin{minipage}{\textwidth}
\begin{lstlisting}[style=customc,label=lst:HXX_IRM,caption=lRM\_NAME.hxx ]
#ifndef _IRM_NAME_HXX
#define _IRM_NAME_HXX

#include "IrradiationModel.hxx"
using namespace std;
class CLASSLogger;
class EvolutionData;
//----------------------------------------------------------------------//
/*!
 Define a IM_NAME
Description
 @author YourName
 @version 3.0
 */
//________________________________________________________________________
class IM_NAME : public IrradiationModel
{
	public :
	IM_NAME(); //constructor

	/*!
	virtual method called to perform the irradiation calculation using a set of cross sections.
	 \param IsotopicVector IV isotopic vector to irradiate
	 \param EvolutionData XSSet set of corss section to use to perform the evolution calculation
	 */
	EvolutionData GenerateEvolutionData(IsotopicVector IV, EvolutionData XSSet, double Power, double cycletime);
	//}
	private :
	//declare your private variables here	
};
#endif
\end{lstlisting}
\end{minipage} 
\end{center}

Open the .cxx file and copy past the following in it (replacing NAME by the same name you used in the .hxx).
\begin{center}
\begin{minipage}{\textwidth}
\begin{lstlisting}[style=customc,label=lst:CXX_IRM,caption=lRM\_NAME.cxx ]
#include "IRM_NAME.hxx"
#include "CLASSLogger.hxx"
#include <TGraph.h>
//Add whatever includes
using namespace std;
//________________________________________________________________________
IRM_NAME::IRM_NAME():IrradiationModel(new CLASSLogger("IRM_NAME.log"))
{
	// do what you want
}
//________________________________________________________________________
EvolutionData IRM_NAME::GenerateEvolutionData(IsotopicVector FreshFuelIV, EvolutionData XSSet, double Power, double cycletime)
{
	EvolutionData GeneratedDB = EvolutionData(GetLog());
	GeneratedDB.SetPower(Power );
	GeneratedDB.SetReactorType(ReactorType );
	
//Your Solver algorithm has to  fill GeneratedDB with the calculated inventories
//using :
GeneratedDB.NucleiInsert(pair<ZAI, TGraph*> (ZAI(Z,A,I), new TGraph(SizeOfpTime, pTime, pZAIQuantity)));	
		
	return GeneratedDB;
}
\end{lstlisting}
\end{minipage} 
\end{center}
The function \textbf{GenerateEvolutionData} returns a \emph{EvolutionData} (see section~\ref{sec:EvolutionData}) containing the inventories evolution over time. This has to be done according to the fresh fuel composition (\textbf{FreshFuelIV}), to the mean cross sections (\textbf{XSSet}), to the (\textbf{Power} : thermal power (W)) and to the irradiation time (\textbf{cycletime} (seconds)).
To fill this  \emph{EvolutionData} you have to call the method  \textbf{NucleiInsert} which associates a nucleus (a ZAI) to  a root object \href{http://root.cern.ch/root/html/TGraph.html}{TGraph}
\footnote{http://root.cern.ch/root/html/TGraph.html}. This TGraph is the evolution (\textbf{pZAIQuantity} in atoms) of this associated nucleus  (\textbf{ZAI(Z,A,I)}) over time (\textbf{pTime} in seconds). This TGraph has \textbf{SizeOfpTime} points.

After making the appropriate changes in this two files to make the function \textbf{GenerateEvolutionData} to return the fuel evolution  (fill free to look at\\
 \$CLASS\_PATH/source/Model/Irradiation/*xx to get inspiration ),  two choices are offered to you. You can compile the two files of your model with your CLASS input or you can add this model to the CLASS package. The second option will modify the CLASS software and we will be no longer able to troubleshoot your scenario. So use the second option only if you are a completely independent user !


\subsection{Compile your Irradiation model with your CLASS executable :}
\begin{center}
\begin{minipage}{\textwidth}
\begin{lstlisting}[style=terminal,label=lst:Compile]
g++ -g -O -I $CLASS_include -L $CLASS_lib -lCLASSpkg `root-config --cflags` 
	`root-config --libs` -fopenmp -lgomp -Wunused-result -c My_MODEL.cxx
		
\rm CLASS* ; g++ -o CLASS_exec MyScenario.cxx My_MODEL.o -I $CLASS_include -L $CLASS_lib -lCLASSpkg `root-config --cflags` `root-config --libs` -fopenmp -lgomp -Wunused-result

\end{lstlisting}
\end{minipage} 
\end{center}

\subsection{Your Irradiation model in the CLASS library :}
Move your  IRM\_NAME.hxx and  IRM\_NAME.cxx in \$CLASS\_PATH/source/Model/Irradiation/. Then open with your favourite text editor the file \\ \$CLASS\_PATH/source/src/Makefile, find "OBJMODEL" and add \$(IM)/IRM\_NAME.o within the others \$(IM) objects. Then re-compile CLASS, fix the compilation errors ;) and voil� your irradiation model is now available in the CLASS library.


