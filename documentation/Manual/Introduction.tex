% !TEX root = USEGUIDE.tex
\part{Introduction}
CLASS is dynamical fuel cycle code. It is a C++ library containing objects and methods to describes all the facilities present in a fuel cycle such as reactor, spent fuel pool, fuel fabrication plant, ...
A  dynamical fuel cycle code simulates the whole reactor fleet and its associated fuel cycle plants and storages. It models the transition from an initial state to a final one (e.g replacement of PWR by FBR-Na, U/Pu to Th/U cycle, phase out,...). The results are mainly isotopic inventories (or related values such as decay heat) and inventories flows (annual heavy metal to reprocessed ...) in each element of the fuel cycle (reactor, fabrication plant, storage ...) over time. 


\part{First Steps}
\chapter{Package Content}
The CLASS package contains the followings :
\begin{itemize}
\item data/ : folder containing nuclei properties
\subitem chart.JEF3T : file containing decay constants and branching ratios
\subitem Mass.dat : file containing molar masses

 \item DATA\_BASES/ : this folder contains decay data base and reactor data bases
 \subitem DECAY/ : decay data base
 \subitem FBR-Na/ : Data base(s) related to Fast reactor
 \subitem PWR/ : Data base(s) related to Pressurised Water Reactor
 
 \item documentation/ : folder containing this user guide an its .tex sources

\item example/ : folder containing simple examples of CLASS input

\item gui/ : folder containing sources of the graphical user interface for CLASS outputs

\item lib/ : folder containing the CLASS library (once compiled)
\item source/ : folder containing CLASS sources
\subitem include/
\subitem Model/ : folder that contain the sources related to the physics models (EquivalenceModel , XSModel and IrradiationModel)
\subitem src/
\item Utils/ : folder containing utility software related to reactor data base generation
\subitem EQM/ : Example of software to generate equivalence model
\subitem MURE2CLASS/ : Software to convert MURE (a fuel depletion code)  output to \hyperref[sec:EvolutionData]{EvolutionData} format
\subitem XSM/ : Software to generate cross section predictor
 
 
\end{itemize}

\chapter{Install procedure}

\section{Requirement}

\begin{center}
\begin{minipage}{\textwidth}
\begin{itemize}
\item User skills : Good knowledge of C++. Abilities in using Root (cern). Experience in depletion codes and neutron transport codes.
\item OS : CLASS is known to work under Linux (64  bits) and MacOSX (64 bits). It  has never been tested on any Windows distribution.
\item Root (CERN) :  
CLASS uses Root to store output data. 
The graphical user interface CLASSGui is based on Root.
Some algorithms uses the TMVA module of Root.
 \item C++ compiler :  we recommend to use a gnu compiler like gcc4.8. 
If your platform is DARWIN (Mackintosh OSX) we strongly recommend not to use the clang compiler\\
You should install macport. then types this following command in terminal :\\
\begin{lstlisting}[style=terminal]
sudo port install gcc48
sudo port select --set gcc mp-gcc48
\end{lstlisting}
\end{itemize}
\end{minipage}
\end{center}

\begin{center}
\line(1,0){250}
\end{center}

\begin{large}
\begin{center}
\textbf{IMPORTANT NOTE : } \\
\end{center}

\end{large} 
The actual root package (version 5.34/20 ) and earlier (and maybe latter) has a memory leak issue when using TMVA leading to a \textbf{freeze of your computer.}
To avoid this dramatical error to happen do the following : \\
If the thread 
\href{http://root.cern.ch/phpBB3/viewtopic.php?f=3&t=18360&p=78586&hilit=TMVA#p78586}{RootTalk}
\footnote{http://root.cern.ch/phpBB3/viewtopic.php?f=3\&t=18360\&p=78586\&hilit=TMVA\#p78586} 
or \href{https://sft.its.cern.ch/jira/browse/ROOT-6551}{RootSupport} \footnote{https://sft.its.cern.ch/jira/browse/ROOT-6551}
indicates status solved then download and install  the more recent ROOT version.\\
If the status is still unresolved proceed as follow : \\
Open with your favourite text editor the file  \$ROOTSYS/tmva/src/Reader.cxx (\$ROOTSYS is the path to your ROOT installation folder) and replace the following :\\

\begin{center}
\begin{minipage}{\textwidth}
\begin{lstlisting}
TMVA::Reader::~Reader( void )
{
   // destructor

   delete fDataSetManager; // DSMTEST

   delete fLogger;
}
\end{lstlisting}
\end{minipage}
\end{center}

by :\\
\begin{center}
\begin{minipage}{\textwidth}
\begin{lstlisting}
TMVA::Reader::~Reader( void )
{
   // destructor
   std::map<TString, IMethod* >::iterator itr;
   for( itr = fMethodMap.begin(); itr != fMethodMap.end(); itr++) {
      delete itr->second;
   }
   fMethodMap.clear();

   delete fDataSetManager; // DSMTEST

   delete fLogger;
}
\end{lstlisting}
\end{minipage}
\end{center}

then type in your terminal : 

\begin{center}
\begin{minipage}{\textwidth}
\begin{lstlisting}[style=terminal]
cd $ROOTSYS
sudo make -j
\end{lstlisting}
\end{minipage}
\end{center}

\begin{center}
\line(1,0){250}
\end{center}


\section{Installation}
Decompress the CLASS.tar.gz in your wanted location. Then, you have to add some environment variables. If your using tcsh  edit the file \$HOME/.tcshrc, and copy past the following  changing \emph{YourPathToCLASS} by the path  of your CLASS installation folder:
\begin{center}
\begin{minipage}{\textwidth}
\begin{lstlisting}
setenv CLASS_PATH YourPathToCLASS
setenv CLASS_lib ${CLASS_PATH}/lib
setenv CLASS_include ${CLASS_PATH}/source/include
setenv PATH ${PATH}:${CLASS_PATH}/bin/gui
\end{lstlisting}
\end{minipage}
\end{center}
 Then type in terminal:
\begin{center}
\begin{minipage}{\textwidth}
\begin{lstlisting}[style=terminal]
source $HOME/.tcshrc
cd $CLASS_PATH/
./instal.sh
\end{lstlisting}
\end{minipage}
\end{center}

It will install CLASS library in \$CLASS\_PATH\/lib, compile the CLASSgui (in gui), and set the correct pathway for the decay base.
One should have in terminal the following lines :
\begin{center}
\begin{minipage}{\textwidth}
\begin{lstlisting}[style=terminal]
********************************
******INSTALLING CLASS V4 ******
********************************
[...]
libCLASSpkg.so done
[...]
libCLASSpkg_root.so done
********************************
[...]
CLASSGui done
CLASSGui is now available in /Users/mouginot/CLASS_Test/CLASS_V4/Gui/bin !!
********************************
Decay base Configuration done!
********************************
**** INSTALLATION COMPLETED ****
********************************
\end{lstlisting}
\end{minipage}
\end{center}


\chapter{CLASS Execution}
CLASS is a set of C++ libraries, there is no CLASS binary file. A CLASS executable has to be build by user using objects and methods defined in the CLASS package. \\
The compilation line for generating your executable from a .cxx file is the following :

\begin{center}
\begin{minipage}{\textwidth}
\begin{lstlisting}[style=terminal]
g++ -o CLASS_exec YourScenario.cxx -I $CLASS_include -L $CLASS_lib -lCLASSpkg `root-config --cflags` `root-config --libs` -fopenmp -lgomp -Wunused-result -lTMVA
\end{lstlisting}
\end{minipage}
\end{center}

\chapter{News, forum, troubleshooting, doxygen ...}
CLASS has a \href{https://forge.in2p3.fr/projects/classforge}{forge}\footnote{https://forge.in2p3.fr/projects/classforge} hosted by the IN2P3  where you can find :

\begin{itemize}
\item \href{https://forge.in2p3.fr/projects/classforge/boards}{A forum}\footnote{https://forge.in2p3.fr/projects/classforge/boards} where you are invited to post your trouble about CLASS installation and usage. You may find the answer to your trouble on a already posted thread.
\item \href{https://forge.in2p3.fr/projects/classforge/embedded/doxygen/inherits.html}{A doxygen}\footnote{https://forge.in2p3.fr/projects/classforge/embedded/doxygen/inherits.html} where all the CLASS objects and methods are defined and explained.
\item \href{https://forge.in2p3.fr/projects/classforge/news}{News}\footnote{https://forge.in2p3.fr/projects/classforge/news} : All the news related to CLASS
\end{itemize}
A \href{classuser-l@ccpntc02.in2p3.fr}{Mailing List}\footnote{classuser-l@ccpntc02.in2p3.fr} also exist in order to be warned of all the change inside CLASS and to allow user to exchange directly on the code. One can join the mailing list through the following  \href{http://listserv.in2p3.fr/cgi-bin/wa?SUBED1=classuser-l&A=1}{link}\footnote{http://listserv.in2p3.fr/cgi-bin/wa?SUBED1=classuser-l\&A=1}.